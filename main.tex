\documentclass[a4paper,10pt]{article}
\usepackage[spanish]{babel}
\usepackage[latin1]{inputenc}
\usepackage{anysize} % Soporte para el comando \marginsize
\usepackage{listings}
\usepackage{formular}
\usepackage[pdftex]{graphicx}
\usepackage{setspace}
\usepackage{booktabs}



\DeclareGraphicsExtensions{.pdf,.png,.jpg}

\usepackage{color}
\definecolor{gray97}{gray}{.97}
\definecolor{gray75}{gray}{.75}
\definecolor{gray45}{gray}{.45}

\lstset{ frame=Ltb,
     framerule=0pt,
     aboveskip=0.5cm,
     framextopmargin=3pt,
     framexbottommargin=3pt,
     framexleftmargin=0.4cm,
     framesep=0pt,
     rulesep=.4pt,
     backgroundcolor=\color{gray97},
     rulesepcolor=\color{black},
     %
     stringstyle=\ttfamily,
     showstringspaces = false,
     basicstyle=\small\ttfamily,
     commentstyle=\color{gray45},
     keywordstyle=\bfseries,
     %
     numbers=left,
     numbersep=15pt,
     numberstyle=\tiny,
     numberfirstline = false,
     breaklines=true,
   }

% minimizar fragmentado de listados
\lstnewenvironment{listing}[1][]
   {\lstset{#1}\pagebreak[0]}{\pagebreak[0]}

\lstdefinestyle{consola}
   {basicstyle=\scriptsize\bf\ttfamily,
    backgroundcolor=\color{gray75},
   }

\lstdefinestyle{C}
   {language=C,
   }


\renewcommand*\lstlistingname{Listado}


\marginsize{3cm}{3cm}{2.5cm}{2.5cm}
\setlength{\parindent}{25pt}

%opening
\title{\textbf{ Wireless communications
in NS3}\\ Simulating wireless communications}
\author{Alejandro Juli\'an Ferro Bejerano}

\begin{document}


\maketitle

\begin{figure}[!h]
\centering
   \includegraphics[width=8cm]{escudo_esi.jpg}
\end{figure}

\newpage

\tableofcontents

\newpage

\section{Objetives}

The main objective of this documents is to understand wireless communications and guide student in
its first simulation process through NS3 simulator.

\singlespacing
To do so, we will answer some fundamental questions. Let's look

\section{Changing the script}
Starting with the original example stats and responds as would the following activities.

\subsection{increment the distances used in each iteration of the simulation to: 25 50 75 100 125 145 147
150 152 155 157 160 162 165 167 170 172 175 177 180 185 190 195 200 210 220 230 240 250
300 350 400 450 500 600 750 1000}

\singlespacing
En concreto, los datos necesarios han sido extra\'idos de la p\'agina http://www.planecrashinfo.com/database.htm.

%(Fig.\ref{fig:inicio}).
        \begin{figure}[h]
        	\centering
    \includegraphics[scale=0.75]{mineria1.png}
    \caption{Extractos de la p\'agina}
    \label{fig:inicio}
        \end{figure}
\pagebreak
En dicho enlace los datos se encuentran ordenados por a\~no y posteriormente aparece una entrada por cada accidente ocurrido en ese a\~no que est\'an ordenados por fecha. Para obtener una base de datos sobre la que podamos trabajar se ha desarrollado un script en Python que obtiene los datos de la p\'agina  http://www.planecrashinfo.com/database.htm usando un agente empleando los conocimientos adquiridos en la asignatura Sistemas Multiagentes.
\singlespacing
	La base de datos resultante incluye todos los accidentes de aviaci\'on, comerciales, militares y privados de todo el mundo que como consecuencia ocasionaron v\'ictimas mortales. Tambi\'en incluye todos los accidentes de carga, "ferry flight" o cabotaje (devolver un avi\'on a la base), posicionamiento y vuelos de prueba en los que han aparecido v\'ictimas mortales.
\singlespacing
	En este enlace se puede comprobar la base de datos extra\'ida
    https://docs.google.com/spreadsheets/d/19JQdbo3O3qIK9q9Bsp7uHB9bMkmrGldf-9spNwW7x1A/pubhtml.
    \singlespacing
    A continuaci\'on se realizar\'a una descripci\'on cuantitativa y cualitativa de la base de datos que se ha obtenido tras el uso del script anteriormente mencionado.

\subsection{Descripci\'on cuantitativa}
La base de datos cuenta con 3.803 entradas y 12 campos, que hacen un total de 45.636 registros.
 \singlespacing
	No se trata de una base de datos con un gran n\'umero de entradas, ya que como se ha mencionado anteriormente los accidentes a\'ereos con v\'ictimas mortales no son sucesos muy comunes. A esto hay que a\~nadirle que la extracci\'on de la base de datos se ha realizado tomando los datos a partir de 1950, por lo que realmente la base de datos original contar\'ia con un mayor n\'umero de entradas. Este aspecto se comentar\'a posteriormente en el apartado de preprocesamiento.

\subsection{Descripci\'on cualitativa}
Una breve definici\'on de la informaci\'on que nos pueden ofrecer los diferentes campos de la base de datos:
\begin{itemize}
\item \textbf{Date:} fecha en la que tiene lugar el accidente con el formato ``Month DD, YYYY''. Ejemplo: ``October 01, 1916 ''.
\item \textbf{Time:} hora local a la que ocurre el accidente en formato ``HH:MM''. Ejemplo: ``18:30''.
\item \textbf{Location:} lugar aproximado  donde tiene lugar el accidente.
\item \textbf{Operator:} Aerol\'inea u operador de la aeronave.
\item \textbf{Flight:} n\'umero de vuelo asignado por el operador de la aeronave.
\item \textbf{Route:} ruta completa o parcial antes de producirse el accidente.
\item \textbf{AC Type:} tipo o modelo de aeronave.
\item \textbf{Registration:} registro de la ICAO (International Civil Aviation Organization).
\item \textbf{cn/ln:} n\'umero construcci\'on o n\'umero de serie / l\'inea o fuselaje.
\item \textbf{Aboard:} total de personas a bordo (pasajeros/tripulaci\'on).
\item \textbf{Fatalities:} total de muertes a bordo (pasajeros/tripulaci\'on).
\item \textbf{Ground:} muertos colaterales del accidente.
\item \textbf{Summary:} descripci\'on del accidente y causa si se conoce.
\end{itemize}

\begin{figure}[h]
        	\centering
    \includegraphics[scale=0.75]{mineria2.png}
    \caption{Representaci\'on accidentes de la base de datos}
    \label{fig:inicio}
        \end{figure}


\section{Hip\'otesis}
Se pretende clasificar cierto vuelo en una categor\'ia de riesgo definida previamente seg\'un su nivel de peligrosidad. Esto se har\'a en base a las caracter\'isticas del vuelo que es objeto de estudio.
\singlespacing
En la clasificaci\'on ser\'a necesario emplear un n\'umero impar de categor\'ias para que exista un elemento central sobre el que pivotar, comprendiendo en el presente caso cinco niveles de mayor a menor peligrosidad. Estas categor\'ias son:
\begin{itemize}
\item A. Ruta extremadamente peligrosa
\item B. Ruta muy peligrosa
\item C. Ruta peligrosa
\item D. Ruta poco peligrosa
\item E. Ruta no peligrosa
\end{itemize}

\pagebreak
	Los par\'ametros que se tendr\'an en cuenta para definir esta medida de riesgo y poder realizar una posterior clasificaci\'on son: hora, origen, destino, escalas, la compa\~n\'ia a\'erea y modelo de avi\'on.
    \singlespacing
	Estas caracter\'isticas del vuelo han sido escogidas por ser las que con mayor precisi\'on especifican los aspectos que pueden afectar a la peligrosidad del vuelo.
    \singlespacing
	El sistema ser\'a de ``Segmentaci\'on'' ya que ante el valor de determinadas entradas (hora, origen, destino, escalas, la compa\~n\'ia a\'erea y modelo de avi\'on) este seleccionar\'a de entre distintas clases predefinidas la clase que m\'as se ajuste.

\section{Metodolog\'ia}
La metodolog\'ia utilizada para abordar el problema, como se ha explicado anteriormente en el apartado de introducci\'on, ser\'a el proceso KDD. Este proceso se compone de cinco fases: selecci\'on de la base de datos, preprocesamiento previo descriptivo, transformaci\'on de la base de datos, miner\'ia de datos e interpretaci\'on de los datos resultantes.

\subsection{Selecci\'on de la base de datos}
Se tomar\'an como vuelos interesantes para el desarrollo del sistema los vuelos que sean comerciales, desechando los vuelos privados y militares. Se desechan estos tipos de vuelo porque no son objeto de estudio, ya que el sistema se centra en el estudio de vuelos comerciales.
\singlespacing
	Como casos representativos solo se utilizar\'an los sucesos producidos a partir de 1950. Tomar como referencia los accidentes a\'ereos de la primera mitad de siglo no es adecuado ya que la tecnolog\'ia empleada en esa \'epoca dista mucho de la actual, a lo que hay que sumarle el contexto de dos guerras mundiales.
    \singlespacing
	Por otro lado, se desechar\'an los campos ``Registration'', ``cn/ln'' y ``Flight'' ya que no son necesarios para priorizar el riesgo del vuelo.

\singlespacing
\begin{table}[htbp]
\centering
\begin{tabular}{p{4cm} p{8cm}}
\hline \hline
Campos eliminados & Descripci\'on\\
\hline \hline
Registration & Es el registro de la ICAO, se ha decidido eliminarlo debido a que no es relevante para estimar el riesgo o peligrosidad de las rutas.\\
\hline
cn/ln & N\'umero de construcci\'on o n\'umero de serie / l\'inea o identificaci\'on del fuselaje. Se desprecier\'an los datos relativos al fuselaje ya que no es uno de los atributos determinantes de riesgo del vuelo.\\
\hline
Flight & El n\'umero de vuelo asignado por el operador de la aeronave, no es interesante ya que cada aerol\'inea tiene su propio sistema de numeraci\'on, es demasiado singular para tenerlo en cuenta.\\
\hline \hline

\end{tabular}
\caption{Campo ``Operator''.}
\label{tabla:autores}
\end{table}

    \singlespacing
	Como resultado de esta selecci\'on se obtiene una tarjeta de datos se obtiene una tarjeta de datos compuesta por vuelos que son comerciales y posteriores a 1950, sin los campos anteriormente mencionados.
    \singlespacing
	La tarjeta de datos resultante cubre razonablemente el espacio de entradas y salidas, ya que con ella manejamos datos como el modelo de avi\'on, operadora, ruta, fecha y hora de los accidentes ocasionados en vuelos comerciales relativamente actuales, y que son suficientes para determinar el nivel de peligrosidad de un vuelo.

\subsection{Preprocesamiento previo descriptivo}
Ante la incertidumbre de algunos factores se debe determinar la completitud de la base de datos, es decir, analizar la carencia, imprecisi\'on e incertidumbre y estimar el nivel de ruido de la base de datos para ver si es tolerable.
\singlespacing
	Se observan algunos registros con el valor ``?'' referentes al campo ``Time''. Se ha decidido tomar como valor por defecto la hora en la que m\'as sucesos tengan lugar del resto de registros de este campo en estos casos. Algunos registros del campo ``Time'' tienes adem\'as de la hora el par\'ametro c, que indica que la hora est\'a en formato Central European Time, y el par\'ametro Z, en este caso el formato estar\'a en Hora Zul\'u; se ha dise\~nado un algoritmo que busque los registros en los que se dan estos casos y se transformen a hora local.
    \singlespacing
	Se eliminan las entradas en las que el campo `AC Type'' es ``?'', ya que no se puede adivinar el modelo del avi\'on al que hace referencia el accidente. Posteriormente y por el mismo motivo tambi\'en se eliminan las entradas en las que el campo `Operator'' es ``?''.
    \singlespacing
	En algunos registros de distintos campos se han detectado errores l\'exicos que causar\'an problemas en el procesamiento posterior de los datos. La lista de errores detectados y la soluci\'on aplicada es la siguiente.

\singlespacing
\begin{table}[htbp]
\centering
\begin{tabular}{p{5cm} p{7cm}}
\hline \hline
Error& Soluci\'on \\
\hline \hline
Pico X, pico Y de Monte Kilimajaro &Monte Kilimajaro\\
\hline
Sept-+le& Sept-\^Iles\\
\hline
Sondrestr÷mfjord& Sondrestrom Fjord\\
\hline
S\'Oo Tom\'U& S\~ao Tom\'e\\
\hline
Reykjav\'Yk& Reykjav\'ik\\
\hline
Crici·ma& Crici\'uma\\
\hline
B¯r Mogre´n& Bir Moghrein\\
\hline
Feij& Feij\'o\\
\hline
Gharda´a& Gharda\"ia\\
\hline
Lbeck&  L\"ubeck\\
\hline
San Jernimo de Moravia& San Jer\'onimo de Moravia\\
\hline \hline
Otros& Se eliminan ``, X mile[s] of Y of Z'' \\
\hline
Otros& Se eliminan ``near '' y ``near of''. \\
\hline
Otros& Se eliminan ``X nm of Y of Z''\\
\hline \hline

\end{tabular}
\caption{Campos ``Location'' y ``Route''.}
\label{tabla:autores}
\end{table}

\begin{table}[htbp]
\centering
\begin{tabular}{p{2cm} p{10cm}}
\hline \hline
Se eliminan ``charter - '' y `` - charter''&\\
\hline
Se eliminan ``- air taxi'' y ``Air Taxi -''&\\
\hline \hline

\end{tabular}
\caption{Campo ``Operator''.}
\label{tabla:autores}
\end{table}

\pagebreak
En aquellos registros de los campos ``Fatalities'' y ``Aboard'' sean desconocidos se tomar\'a la media de los accidentes que usen el mismo modelo, si s\'olo es desconocidos ``Fatalities'' se tomar\'an como muertos el total de a bordo.
    \singlespacing
No es normal que haya fallecidos en el suelo. En caso de encontrar ``?'' se le dar\'a un valor de ``0''.
    \singlespacing
Si en el campo ``Route'' aparece un registro con el valor ``?'' se utilizar\'a como valor por defecto el valor del campo ``Location'' de la misma entrada. Lo mismo ocurrir\'a en sentido inverso.
\singlespacing
	Los valores del campo ``Summary'' que sean ``?'' se cambiar\'a por ``None''. El campo ``Summary'' es una descripci\'on en lenguaje natural. Se prev\'e hacer text mining de este campo para extraer cierta informaci\'on que pueda ser \'util, como fallos de motor o problema por factor humano.

\subsection{Transformaci\'on}
Al contar con una base de datos relativamente estrecha en cuanto al n\'umero de campos, se dispone a transformar la base de datos pasando de 12 campos a 17 en la base de datos final.
\singlespacing
En los campos ``Fatalities'' y ``Aboard'' se distingue entre tripulaci\'on y resto de pasajeros. Se separar\'an en ``Fatalities\_Passengers'', ``Fatalities\_Crew'', ``Aboard\_Passengers'' y ``Aboard\_Crew'' para poder trabajar con estos datos por separado. A continuaci\'on se muestran unas tablas que detallan los campos introducidos y los campos eliminados.
\singlespacing
\begin{table}[htbp]
\centering
\begin{tabular}{p{3cm} p{7cm}}
\hline \hline
Campos eliminados& Descripci\'on\\
\hline \hline
Fatalities &Total de muertes a bordo, se sustituir\'a por dos nuevos campos, Fatalities\_Passengers y Fatalities\_Crew\\
\hline
Abordo&Total de personas a bordo,ser\'a sustituido por Aboard\_Passengers y Aboard\_Crew\\
\hline
Route & Indica la ruta completa o parcial antes de producirse el accidente, incluyendo las escalas. Se sustituir\'a por los campos Origen, Destino, y Escalas\\
\hline \hline
\end{tabular}
\caption{Campos eliminados.}
\label{tabla:autores}
\end{table}
\pagebreak
\begin{table}[htbp]
\centering
\begin{tabular}{p{5cm} p{5cm}}
\hline \hline
Campos introducidos &Descripci\'on\\
\hline \hline
\hline
Fatalities\_Passengers &N\'umero total de pasajeros fallecidos\\
\hline
Fatalities\_Crew&N\'umero total de tripulaci\'on fallecida\\
\hline
Aboard\_Passengers & N\'umero total de pasajeros a bordo\\
\hline
Aboard\_Crew &N\'umero otal de tripulaci\'on a bordo\\
\hline
Origen&El primer elemento del campo ``Route''\\
\hline
Destino& El \'ultimo elemento del campo ``Route''\\
\hline
Escalas & Los elementos intermedios entre Origen y Destino del campo ``Route'' o ´´ '' si no existe ninguno\\
\hline
Latitude/Longitud\_Location & Latitud  y longitud del lugar del accidente\\
\hline
Latitude/Longitude\_Origen&Latitud  y longitud del lugar del origen del vuelo\\
\hline
Latitude/Longitude\_Destino&Latitud  y longitud del lugar del destino del vuelo\\
\hline \hline
\end{tabular}
\caption{Campos eliminados.}
\label{tabla:autores}
\end{table}

	Con la localizaci\'on de los accidentes se ha implementado un algoritmo en el que dado el nombre de la ciudad proporcione su latitud y su longitud. Como caso especial, en los registros en los que la localizaci\'on del accidente se indique que ha tenido lugar en un aeropuerto se pondr\'a la longitud y latitud de dicho aeropuerto, no de la ciudad.
\singlespacing
Este algoritmo obtiene, para una campo de texto que  describe el nombre de la localizaci\'on, las coordenadas geogr\'aficas.
\singlespacing
Esta idea surge al observar que la localizaci\'on v\'ia texto es insuficiente para poder procesar de forma correcta los accidentes, ya que no es posible agruparlos (no es posible conocer la ubicaci\'on geogr\'afica simplemente por un campo de texto, es necesario un campo num\'erico, ya que se pueden comparar y relacionar de forma mas f\'acil). El algoritmo hace uso del servicio web de Google Maps, haciendo uso de su versi\'on mas reciente (API v3).
\singlespacing
Por otro lado, uno de los principales problemas observados al realizar el algoritmo es su numero limitado de peticiones diarias, situando este numero en 3000 peticiones por d\'ia, y dado que cada campo tiene 3 localizaciones, solo nos permite procesar 1000 accidentes en 24 horas, algo que es totalmente insuficiente y retrasa en gran medida nuestro trabajo. Debido a este motivo y a optimizaron del uso de la red, se decide implementar un cache de direcciones,  que nos permite almacenar las localizaciones ya consultadas, y obtenerlas de forma r\'apida y sin consumo de peticiones. Con este algoritmo, podemos obtener una ventaja clara, ya que ademas de funcionar mas r\'apido, solo consumimos una petici\'on por localizaci\'on, y dado que un gran numero localizaciones de repiten  2 o mas veces, hace que podamos duplicar el poder de procesamiento del que dispon\'iamos con anterioridad.
\singlespacing
\pagebreak
Antes de implementar el algoritmo, dispon\'iamos de mas de 3000 entradas en nuestra base de datos, por lo que hubi\'eramos necesitado de 3 d\'ias para poder aplicar el algoritmo de forma correcta sobre la totalidad de los datos. Gracias a las mejoras en la implementaci\'on, en un solo d\'ia pudimos obtener la totalidad de los datos.
\singlespacing
Tras cierta deliberaci\'on, tambi\'en observamos las diferencias horarias como un problema. Esto se debe a que los datos proporcionados est\'an en diferentes husos horarios, o incluso en diferentes formatos. Es necesario establecer un formato fijo, as\'i como un huso horario com\'un a todos los casos. Decidimos hacer uso de la hora zul\'u, ya que es inmutable, y no se ve variada a lo largo del tiempo, al contrario que el CET (Horario central europeo) o los husos horarios.
\singlespacing
Mediante la API de Google TimeZone versi\'on 3, realizamos los mismos pasos que en el caso anterior. Tras cada consulta a la API obtenemos la diferencia horaria respecto al tiempo de referencia fijado, lo que nos permite obtener, mediante un calculo sencillo, la hora zul\'u en la que el accidente ocurri\'o. Como posible mejora en este algoritmo establecemos que es posible ademas calcular, mediante la fecha, si la localizaci\'on se sit\'ua en horario de verano o invierno, pero observamos que no todos los pa\'ises cambian de hora entre invierno y verano. El error m\'aximo que cometemos es de 1 hora.
\singlespacing
El procesamiento de los datos se realiza por pasos, ya que primero obtenemos todos los datos (incluido las posiciones geogr\'aficas) y tras la finalizaci\'on de la obtenci\'on, volvimos a calculamos las horas finales. Esto se debe a que Google Timezone API tambi\'en tiene un limite diario de 3000 peticiones, por lo que fue necesario volver a realizar el cache de datos, y procesarlo al siguiente dia, ya que superaba el numero m\'aximo de peticiones.
\singlespacing
Al inicio del proyecto se plante\'o la posibilidad de hacer text mining con el campo ``Summary'' debido a que este registro, que aparece como breve p\'arrafo descriptivo del accidente en la base de datos, contempla datos importantes a la hora de analizar la causa de los accidentes.
\singlespacing
Se opt\'o por un enfoque estad\'istico, utilizando librer\'ias espec\'ificas para el an\'alisis de texto. Paquetes como ``Python Textmining Package'' sirvieron para crear una matriz de t\'erminos contenidos en toda la base de datos. Una vez recopilados los t\'erminos en la matriz se pretend\'ia efectuar un an\'alisis estad\'istico sobre ella para obtener las estructuras y palabras m\'as representativas de cada descripci\'on.
\singlespacing
No obstante, no fue posible encontrar un procedimiento adecuado con el que dotar de significado a las expresiones resultantes. Concluimos que era necesario realizar un an\'alisis sem\'antico mucho m\'as profundo y preciso con el que establecer un contexto en el que encuadrar las expresiones y as\'i poder realizar afirmaciones sobre el contenido del texto. En muchos casos, adem\'as se hac\'ia necesario realizar deducciones a partir de dichas  afirmaciones para obtener conclusiones correctas de cada descripci\'on textual del suceso.
\singlespacing
Debido a los l\'imites temporales del proyecto y a la falta de experiencia del equipo en el campo del an\'alisis natural del leguaje, no se pudo llevar a cabo finalmente la extracci\'on de datos utilizables a posteriori en el an\'alisis de las causas de los accidentes a\'ereos.
\pagebreak

\subsection{Mineria de Datos}
El primer paso consiste en escoger la funci\'on de miner\'ia de datos adecuada para el prop\'osito del proyecto. Se pretende utilizar clustering jer\'arquico para clasificar las distintas categor\'ias.


\subsubsection{Funci\'on de Similitud}
Tras el preprocesamiento de la base de datos ya se han seleccionado las variables que se tendr\'an en cuenta para la clasificaci\'on.
Para clasificar adecuadamente los accidentes a\'ereos que componen la base de datos es necesario determinar lo similares que son entre ellos, para ello se ha definido una funci\'on de similitud.

Los atributos sobre los que se aplicar\'a la funci\'on de similitud son:
\begin{itemize}
\item Fecha
\item Hora
\item Coordenadas lugar del accidente
\item Coordenadas origen de la ruta
\item Coordenadas destino de la ruta
\item Escalas
\item Modelo
\item Pasajeros a bordo
\item Tripulaci\'on a bordo
\item Pasajeros fallecidos
\item Tripulaci\'on fallecido
\item Fallecidos en tierra
\end{itemize}

\singlespacing
Se pretende comparar atributo con atributo entre todos los accidentes seleccionados para la tarjeta de datos, de forma que en una primera aproximaci\'on se obtendr\'a un vector de dimensi\'on 12 para cada par de accidentes comparado.
\singlespacing
Tras este paso, se realiza una normalizaci\'on de los distintos atributos y adem\'as se aplicaron distintos pesos a dichos atributos, seg\'un la importancia que subjetivamente se ha considerado.
\singlespacing
Por \'ultimo se calcula la media para obtener un valor \'unico que ser\'a el que represente la similitud  entre cada par de accidentes.
\singlespacing
A continuaci\'on se describe la funci\'on de similitud definida:
\singlespacing
La funci\'on de similitud consta de cuatro subfunciones definidas espec\'ificamente para los atributos seleccionados. Estas son:
\begin{itemize}
\item \textbf{diferenciaFechas()}: Se le pasan por par\'ametro tres atributos, dos cadenas que representan la fecha con el formato ``dd/MM/yyyy'', y el \'ultimo par\'ametro es ``valor'', y depende del valor que se le otorgue devolver\'a el resultado en un formato u otro. Para calcular la distancia entre fechas se ha decidido descartar la distancia anual, es decir, s\'olo se tendr\'an en cuenta los d\'ias y los meses, para ello se establece que a la hora de calcular la distancia el a\~no de ambas fechas sea el mismo.

\item \textbf{diferenciaHoras():} La estructura de esta subfunci\'on es similar a la que acaba de ser descrita, pero el formato de los dos primeros atributos es ``hh:mm''. Se han definido funciones auxiliares que ayudan a obtener el resultado final deseado, algunas de ellas son diferenciaHorasDias,  o cantidadTotalHoras.

\item \textbf{euclidean():} Se usa con los atributos num\'ericos de los fallecidos y personas a bordo, y las coordenadas. Han sido definidas dos funciones eucl\'ideas para calcular esta distancia con elementos de una o dos dimensiones respectivamente. A continuaci\'on de muestra el c\'odigo de ambas:
\begin{lstlisting}[style=C]
public static double euclidean(double[] x, double[] y){
	        double sum = Math.pow(y[0]-x[0],2) + Math.pow(y[1]-x[1],2);
	        return Math.sqrt(sum);
	}
	public static double euclidean(int x, int y){
	        double sum = Math.pow(x - y, 2);
	        return Math.sqrt(sum);
	}

\end{lstlisting}
\item \textbf{comparar():} Es usada con los campos tipo ``String'', escalas, y modelo, devuelve el valor ``1'' cuando dos valores son iguales y ``0'' cuando son distintos.

\begin{lstlisting}[style=C]
	public static int comparar(String s1, String s2){
			if(s1.equals(s2))
				return 1;
			else return 0;
	}
    \end{lstlisting}
\end{itemize}

\singlespacing
Por otro lado, los pesos asignados a los atributos para calcular la media quedaron de la siguiente manera:
\begin{itemize}
\item Los campos ``Pasajeros a bordo'', ``Tripulaci\'on a bordo'', ``Pasajeros fallecidos'', ``Tripulaci\'on fallecido'' y ``Fallecidos en tierra'' se mantuvieron con peso ``1''.

\item A ``Fecha'', ``Hora'', ``Escalas'' y ``Modelo'' se decidi\'o asignarles peso ``2''.

\item Finalmente, a ``Coordenadas lugar del accidente'', ``Coordenadas origen de la ruta'' y ``Coordenadas destino de la ruta'' se les asigna peso ``3''.

\end{itemize}
\pagebreak
\section{Futuros proyectos}
El principal objetivo es aplicar de forma m\'as precisa los algoritmos de text minining que sean de utilidad para obtener la informaci\'on del campo ``Summary'' que puede ser muy relevante a la hora de categorizar.
\singlespacing
Por otra parte se desarrollar\'a un sistema que consuma la informaci\'on a partir de los resultados obtenidos en el proyecto. La aplicaci\'on se encargar\'a de evaluar un vuelo previamente introducido y clasificarlo dentro de un nivel de peligrosidad seg\'un sus caracter\'isticas. La aplicaci\'on deber\'ia ser c\'omoda de usar por cualquier usuario medio, por tanto ser\'ia conveniente desarrollar una interfaz de usuario sencilla e intuitiva.
\singlespacing
Tambi\'en ser\'ia interesante cruzar la base de datos de accidentes con otra base de datos referente a sucesos climatol\'ogicos para poder ampliar el conocimiento sobre la causalidad de los accidentes y poder a\~nadir m\'as par\'ametros al an\'alisis de riesgo.

\section{Conclusi\'on}
 En la asignatura se pretende desarrollar un sistema completo basado en el proceso KDD, lo cual ha sido plasmado en el presente trabajo. Se han aplicado los conocimientos de teor\'ia pertenecientes a la asignatura, siguiendo el proceso estudiado en clase. Se han adquirido las competencias necesarias para obtener conocimiento a partir de grandes vol\'umenes de datos.
Se ha puesto de manifiesto que no se trata de un proceso trivial, ya que no se usa la  metodolog\'ia convencional. \singlespacing
Adem\'as existe la dificultad a\~nadida de la necesidad en algunos casos de utilizar conocimientos transversales a la hora de sintetizar los datos. Es necesario tambi\'en conocer el \'ambito del problema y las particularidades asociadas al mismo.

\section{Webgraf\'ia y herramientas}
\begin{itemize}
\item https://www.nsnam.org/doxygen/
\item \LaTeX
\end{itemize}



\end{document}
